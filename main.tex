\documentclass[a4paper]{article}
\usepackage[utf8]{inputenc}
\usepackage[margin=2cm]{geometry}
\usepackage{epsdice}

\begin{document}

%\begin{titlepage}
\begin{center}
    \Huge \textsc{Canoe}\\
    \LARGE Designed by Bruce Alsip\\
    \large PDF by Niko Lepka\\
    $v.2022.7.27.0$
\end{center}
%\end{titlepage}

\section{Introduction}
Canoe is a strategy game for two players, which can be described as a sort of mix between Checkers and Backgammon.

The object of the game is to capture and bear off as many pieces as possible before the end of the game.
Whichever player has the most points by the end is the winner.

\paragraph{Note} In lieu of not having access to the actual rules, this write-up has been compiled from the example videos available on the website \texttt{http://canoegame.com}.

\section{Components}
\begin{itemize}
    \item 1 Canoe board
    \item 7 Black play cubes
    \item 7 Ivory play cubes
    \item 2 Black dice
    \item 2 Ivory dice
\end{itemize}

\section{Terminology}
\begin{itemize}
    \item \textbf{Bank} -- The area of the board in which the cubes start off.
    \item \textbf{Bearing Off} -- Transporting a cube off of the board for the purpose of scoring.
    \item \textbf{Cube} -- Cube shaped game piece with the numbers 1, 8, 16, 24, 32, 40 on its sides. Not a die.
    \item \textbf{Die} -- Cubes with dots on them, used to determine movement.
    \item \textbf{Dice} -- Plural of \textit{Die}.
    \item \textbf{Pocket} -- The area where a player keeps their pinched and borne-off cubes.
    \item \textbf{Set} -- Two cubes of like number stacked on top of each other.
\end{itemize}

\section{Setup}
\begin{enumerate}
    \item Players decide amongst themselves which colour they wish to play.
    \item Each player takes six of the seven available cubes and mixes them up, rolling them onto the play area.
    \item These cubes are then placed in the bank behind the player's starting line in whichever configuration they see fit.
    \item Both players now roll the final cube and place it in the middle of the row furthest away from their cubes.
\end{enumerate}
The player who rolled the highest number on this cube starts the game.

\begin{figure}[!h]
    \centering
    %\includegraphics{}
    Figure goes here
    \caption{Example Setup}
    \label{fig:setup}
\end{figure}

\section{Gameplay}
Players roll two dice of their chosen colour at the start of their turn.
The values of these dice are not combined, but instead used to move one or two cubes around the board individually\footnote{If you've played Backgammon, this should be familiar.}.
\subsection{Moving a Cube}
Before moving a cube, the cube must first be rotated to a different face.
This rotation is entirely based on whether the die rolled was even or odd.
After the cube has been rotated, it is then moved the indicated number of spaces orthogonally in any direction on the board, though never crossing back over the starting line.

Cubes also cannot pass over each other.

\paragraph{If the die shows even} The cube's face value is increased to the next available value
$$1 \to 8 \to 16 \to 24 \to 32 \to 40 \to 1$$
\paragraph{If the die shows odd} The cube's face value is decreased to the previous value
$$40 \to 32 \to 24 \to 16 \to 8 \to 1 \to 40$$
\paragraph{Note} that in either case, a roll-over happens if you try to increase or decrease past the highest or lowest value respectively.

\paragraph{Example} Billy rolled a \epsdice{5} and a \epsdice{2}, and has decided to move his 24 cube five spaces. He first rotates the cube so it shows 16 before moving it those five spaces. Next he wants to move his 1 cube two spaces, he first rotates the cube so it instead shows 8, and then moves it two spaces.

\subsubsection{Stimy}
In the event that a player has exactly two cubes left, and both share the same number and are adjacent, then that player may choose to call ``Stimy'' and only roll one die on their turn.

\subsection{Pinching}
Pinching is when a player moves their cube onto an opponent's cube of equal value.

That is to say, suppose Black has a 32 cube, and Ivory has a 24 cube four spaces away. 
Should Ivory roll a four, they'd be able to pinch Black's cube by first increasing the value up from 24 to 32, then moving the cube four spaces onto Black's cube.

Pinched cubes are removed from play and placed in the pincher's pocket.

\paragraph{Note} It is not allowed on the first move by the first player.

\paragraph{Note} You cannot pinch with a cube that just crossed the starting line.

\subsection{Bearing Off}
To bear off, you must first make a \textit{Set} of two cubes.
Sets are special because they break the normal movement conventions:
\begin{enumerate}
    \item Sets do not rotate before moving. If the cubes show 16, then they'll show 16 even after moving.
    \item Sets can jump over \textbf{any number} of adjacent other cubes or sets as part of their movement. This jump counts as just one move. Single cubes \textbf{cannot} jump.
    \item When moving sets you can \textit{add together} the results of both dice. This is not possible when moving cubes.
    \item Sets cannot pinch, only jump.
    \item Sets must move directly towards the pin on the board, and are not allowed to meander.
\end{enumerate}
To then bear off, a player must then move the stack of cubes over the pin (black dot) on the starting line and into the bank, before then having to move towards the leather patch and over the line on the backmost row.

\paragraph{Note} You cannot make a set with a cube that's just left the bank.

\paragraph{Note} To bear off, a player needs the exact number or number combination to make it to the leather patch. If the roll isn't an exact match, it won't work.

\subsubsection{Canoe}
The first time a player bears off their cubes, \textit{both} of the cubes are pocketed for scoring.

All subsequent times, only one cube is pocketed, while the other is discarded, never to be seen again.

\subsubsection{Seventh Cube}
Once all but the last cube have been pinched or borne off, the seventh cube can then be borne off all by itself.
Doing so successfully sweeps the opponent's remaining cubes into the pocket as points.

Like a set, the seventh cube can also combine the results of the two dice for better reach.
Also like a set, the seventh cube must bear off as soon as possible.

\subsection{Sweeping}
Sweeping happens at the end of the game when bearing off your seventh and final cube.
Successfully doing so sweeps the board, giving you the other player's remaining cubes as points.

\section{Scoring}
Each cube scores its face value, except for the 1, which scores no points.

\end{document}
