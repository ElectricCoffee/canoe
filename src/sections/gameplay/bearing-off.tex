\subsection{Bearing Off}\label{sec:bearing-off}
To bear off, you must first make a \textit{Set} of two cubes.
Sets are special because they break the normal movement conventions:
\begin{enumerate}
    \item Sets do not rotate before moving. If the cubes show 16, then they'll show 16 even after moving.
    \item Sets can jump over \textbf{any number} of adjacent other cubes or sets as part of their movement. This jump counts as just one move. Single cubes \textbf{cannot} jump.
    \item When moving sets you can \textit{add together} the results of both dice. This is not possible when moving cubes.
    \item Sets cannot pinch, only jump.
    \item Sets must move directly towards the pin on the board, and are not allowed to meander.
\end{enumerate}
To then bear off, a player must then move the stack of cubes over the pin (black dot) on the starting line and into the bank, before then having to move towards the leather patch and over the line on the backmost row.

\paragraph{Note} You cannot make a set with a cube that's just left the bank.

\paragraph{Note} To bear off, a player needs the exact number or number combination to make it to the leather patch. If the roll isn't an exact match, it won't work.

\subsubsection{Canoe}
The first time a player bears off their cubes, \textit{both} of the cubes are pocketed for scoring.

All subsequent times, only one cube is pocketed, while the other is discarded, never to be seen again.

\subsubsection{Seventh Cube}
Once all but the last cube have been pinched or borne off, the seventh cube can then be borne off all by itself.
Doing so successfully sweeps the opponent's remaining cubes into the pocket as points.

Like a set, the seventh cube can also combine the results of the two dice for better reach.
Also like a set, the seventh cube must bear off as soon as possible.
