\subsection{Sets \& Bearing Off}\label{sec:bearing-off}
To bear off, a player must first make a \textit{Set} by stacking two of their own cubes of equal value, similar to Pinching.

Sets are special because they break the normal movement rules:
\begin{enumerate}
    \item Sets do not rotate before moving. If the cubes show 16, then they'll show 16 even after moving.
    \item Sets can be formed and moved in the same turn.
    \item Sets can jump over \textbf{any number} of adjacent other cubes or sets as part of their movement (see Figure~\ref{fig:jump}). This jump counts as just one move. Single cubes \textbf{cannot} jump.
    \item When moving sets players can \textit{add together} the results of both dice. This is not possible when moving cubes.
    \item Sets cannot pinch, only jump.
    \item Sets also cannot \textit{be} pinched. They are safe.
    \item Sets must move directly towards the pin on the board using the shortest path possible.
\end{enumerate}
\begin{figure}[!h]
    \centering
    \includegraphics[width=8cm]{../graphics/jump}
    \caption{Example Jump}
    \label{fig:jump}
\end{figure}
To then bear off, a player must move the stack of cubes over the pin (black dot) on the starting line and into the bank, before then having to move towards the leather patch and over the line on the backmost row.

Bearing off requires rolling the exact required number. If the cube is 3 spaces from the line, then a roll of \epsdice{3} or a roll of \epsdice{2} and \epsdice{1} are required to get over the line.

\paragraph{Jumping Example}
Figure~\ref{fig:jump} illustrates an example jump.
Black rolled \epsdice{3} and \epsdice{5}, and wishes to move their stack of 16s.
To get into the bank as quickly as possible, Black uses their \epsdice{3} and makes use of stacks' ability to jump over other pieces by first moving one down, then all the way over all of Ivory's pieces as the second move, only to end up inside the bank with a third move.

\note Sets cannot be formed on a player's first turn.

\note Sets cannot be formed with cubes that have just left the bank.

\note The rule about getting across the line rolling the exact number specifically means doing so in the fewest number of steps. Players are not allowed to artificially extend the path by moving in a zig-zag formation within the bank.

\subsubsection{Canoe}
The first time a player bears off their cubes, \textit{both} of the cubes are pocketed for scoring.

All subsequent times, only one cube is pocketed, while the other is discarded, never to be seen again.

\subsubsection{Seventh Cube}
Once all but the last cube have been pinched or borne off, the seventh cube can then be borne off all by itself.
Doing so successfully sweeps the opponent's remaining cubes into the pocket as points.

Like a set, the seventh cube can also combine the results of the two dice for better reach.
Also like a set, the seventh cube must bear off as soon as possible.

\note Because the seventh cube is a cube and not a Set, it must be rotated once for each of the dice used to move it. Suppose Alice rolled a \epsdice{6} and a \epsdice{4}, and wanted to move her cube showing 40 ten spaces, she would then have to rotate it down to 32 for the first move, and then down to 24 for the second.