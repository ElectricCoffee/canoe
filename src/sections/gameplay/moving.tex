\subsection{Moving a Cube}
Before moving a cube, the cube must first be rotated to a different face.
This rotation is entirely based on whether the die rolled was even or odd.
After the cube has been rotated, it is then moved the indicated number of spaces orthogonally in any direction on the board, though never crossing back over the starting line.

Cubes also cannot pass over each other.

\paragraph{If the Die Shows Even} The cube's face value is increased to the next available value
$$1 \to 8 \to 16 \to 24 \to 32 \to 40 \to 1$$
\paragraph{If the Die Shows Odd} The cube's face value is decreased to the previous value
$$40 \to 32 \to 24 \to 16 \to 8 \to 1 \to 40$$

\paragraph{Note} In either case, a roll-over happens if you try to increase or decrease past the highest or lowest value respectively.

\paragraph{Example} Billy rolled a \epsdice{5} and a \epsdice{2}, and has decided to move his 24 cube five spaces. He first rotates the cube so it shows 16 before moving it those five spaces. Next he wants to move his 1 cube two spaces, he first rotates the cube so it instead shows 8, and then moves it two spaces.

\subsubsection{Stimy}
In the event that a player has exactly two cubes left, and both share the same number and are adjacent, then that player may choose to call ``Stimy''\footnote{The audio quality in the video wasn't very good. The word may actually be ``Stiny'' with an N.} and only roll one die on their turn.